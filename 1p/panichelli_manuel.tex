% !TeX root = ./panichelli_manuel.tex

\documentclass[12pt, a4paper]{report}
\usepackage[utf8]{inputenc} 
%\usepackage[spanish]{babel} 

% Para hipervinculos en las referencias
\usepackage{hyperref}

\usepackage{amsmath,amsfonts,amssymb,amsthm,epsfig,epstopdf,titling,url,array}
\usepackage{fullpage}
\usepackage{fancyhdr}

% Teoremas, corolarios, etc.
% https://www.overleaf.com/learn/latex/theorems_and_proofs
\theoremstyle{definition} % Para que no salga en italicas
\newtheorem{theorem}{Teorema}
\newtheorem{lemma}{Lema}

\pagestyle{fancyplain}
\fancyhf{}

\renewcommand{\headrulewidth}{0pt}

\rhead{Manuel Panichelli}
\chead{1er Parcial AED3: Parte Domiciliaria}
\lhead{L.U. 72/18}

% Distancia del header (\chead) al texto
%\setlength{\headsep}{3pt}

\title{1er Parcial AED3: Parte Domiciliaria}
\author{Manuel Panichelli, 72/18}
\date{\today}

\begin{document}

\maketitle
\newpage

\chapter*{Preámbulo}

Teoremas / Ejercicios utilizados.

\section{Arboles}

\begin{theorem}[Definiciones equivalentes de árbol]\label{teo:tree-equiv}
    Dado $G = (V, X)$ un grafo, las siguientes son equivalentes:

    \begin{enumerate}
        \item G es un árbol, un grafo conexo sin circuitos simples.
        \item G es un grafo sin circuitos simples, y $e$ arista tq $e \notin X$,
        el grafo $G + e$ tiene exactamente un circuito simple el cual contiene a
        $e$.\label{teo:tree-equiv-circ}
        \item G es conexo, pero si se saca cualquier arista queda un grafo no
        conexo. Es decir, toda arista es puente.\label{teo:tree-equiv-puentes}
    \end{enumerate}
\end{theorem}

\begin{lemma}\label{lema:tree-g-e}
    Sea $G = (V, X)$ conexo y $e \in X$.

    \[ G - e \ \text{es conexo} \iff \text{e pertenece a un circuito simple de G} \]
\end{lemma}

\begin{theorem}\label{teo:tree-t-e-f}
    Sean $T = (V, X_T)$ un AG de $G = (V, X)$, $e \in X \setminus X_T$. Luego $T + e - f$ con $f$ una arista del único circuito de $G+e$ es AG de $G$.
\end{theorem}

\chapter*{Resolución}

\section*{Ejercicio 1}

\section*{Ejercicio 2}

\section*{Ejercicio 3}

\section*{Ejercicio 4}

\textit{Un \textbf{puente} de un grafo es un eje del grafo tal que al removerlo
se obtiene un grafo con más c.cs. Sea $G = (V, X)$ un grafo conexo y $e \in X$.
Demostrar que e es un puente de G sii e pertenece a todo AG de G.}

\begin{proof}[Dem]
    Demostremos la ida y la vuelta.

    \quad

    $\Rightarrow$)
    Sea $e$ puente de G. qvq pertenece a todo AG de G.\\
    Va por el absurdo: supongo que no pertenece a ningún árbol generador de G, y sea $T = (V, X_T)$ uno cualquiera. Como $e \notin T$, por (\ref{teo:tree-equiv}.\ref{teo:tree-equiv-circ}), $T + e$ tiene exactamente un circuito simple $C$, el cual contiene a e. Sea $f \in C, f \in X_T$. Como $f$ pertenece al único circuito de $T + e$, por (\ref{teo:tree-t-e-f}) el árbol $T' = T + e - f$ es árbol generador de G, y $e \in T'$. Pero estabamos suponiendo que $e$ no pertenecía a ningún AG de G. Abs! Entonces pertenece a todos.

    \quad

    $\Leftarrow$)
    Como $e$ pertenece a todo AG de G, y por
    (\ref{teo:tree-equiv}.\ref{teo:tree-equiv-puentes}) todas las aristas de un
    arbol son puente, en particular $e$ es puente de G.

\end{proof}





\section*{Ejercicio 5}

\end{document}